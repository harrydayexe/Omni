\begin{abstract}
The creation of Omni, a scalable social media platform made especially for junior software engineers to share personal projects and create professional portfolios, is examined in this dissertation.
The project aims to develop a system that can support viral user growth to millions of users with little performance degradation, acknowledging the shortcomings of current platforms like LinkedIn in catering to this demographic.
Omni uses a Kubernetes-enabled microservices architecture, with a GoLang-developed backend and a frontend that makes use of Go templates, HTMX, and TailwindCSS.
The platform's data management approach is supported by a horizontally scalable MariaDB database that makes use of Snowflake IDs for effective sharding.
The platform's capacity to adapt to growing demand was demonstrated by integrating observability through structured logging and metrics collection and validating scalability through load testing in a Kubernetes environment.
User surveys were used to assess the platform's usability and suitability for the intended audience.
In order to further improve platform performance and user experience, the project ends with a reflection on future work that includes the addition of recommendation algorithms, real-time event streaming, and caching mechanisms.
\end{abstract}
