\chapter{Background}
\label{cha:background}

\section{The Growth of Social Media}
In recent years, social media has become a cornerstone of modern life with 2.19 billion active users on Facebook in 2018 \citep{saha2019impact}. Emerging adults spend approximately 6 hours using social media daily and frequently use multiple platforms simultaneously \citep{vannucci2019use}.
In contemporary life, ``we draw on the benefits of the ubiquitous networks which provide unparalleled opportunities of economic as well as cultural growth'' \citep{hruska2020use}.
Some social media platforms cater to specific demographics, such as LinkedIn, which targets professionals across every industry.
Despite being a professional network, LinkedIn falls flat at serving the needs of unique demographics, such as young software engineers and computer science students.
These demographics do not actively engage with professional social networks as their older counterparts \citep{florenthal2012college}. 

\subsection{A New Professional Network for Junior Software Engineers}
Junior software engineers should focus on growing their professional networks and their portfolio of personal projects to best position themselves in a competitive post-COVID job market, with open roles for software engineers down 34\% in 2025 from 2020 \citep{fredsoftware2025}

LinkedIn is motivated to cater to its largest demographic, which, in turn, discourages younger demographics from engaging on the platform.
While highly motivated to break into the software industry, this demographic does not engage with the traditional professional networking giant.

\subsection{Scalability in Social Network}
According to Chakradhar and Raghunathan, scalable computing systems maintain consistent performance under an expanded workload.
Social networks must maintain high performance in user experience while experiencing rapid growth in their user base \citep{pujol2010little}.
For example, Twitter experienced a rapid growth in active users of over 1000\% between 2008 and 2009, which resulted in significant downtime due to the architecture not being designed to scale efficiently at that rate \citep{sakaki2010earthquake}.
Scalability should be a fundamental philosophy from the inception of a design.
